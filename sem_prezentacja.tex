\input{preamble}


\begin{document}
% \usebackgroundtemplate{\includegraphics[width=\paperwidth]{img/bg.jpg}}
\frame{\titlepage}

% Przebieg prezentacji
\begin{frame}
  \frametitle{Przebieg prezentacji}
  \tableofcontents
\end{frame}

% The following causes the table of contents to be shown 
% at the beginning of every subsection. Delete this, if you do not want it.
\AtBeginSubsection[]{
  \frame<beamer>{
    \frametitle{Przebieg prezentacji}
    \tableofcontents[currentsection,currentsubsection]
  }
}

\section{Wprowadzenie}
\subsection{Dlaczego odwrócone wahadło}
\begin{frame}{Dlaczego odwrócone wahadło?}
	\only<1>{
		\begin{figure}[!htp]
			\includegraphics[width=0.5\textwidth]{img/balans1}
			\caption{Robot balansujący ,,KOSMOS'' zbudowany w Kole Naukowym Robotyków KoNaR
			na wydziale Elektroniki Politechniki Wrocławskiej}
		\end{figure}
	}
	\only<2->{
		\begin{itemize}
			\item obiekt nieliniowy.
			\item trudność w sterowaniu.
			\item Cieszy się dużym zainteresowaniem.
		\end{itemize}
	}
\end{frame}

\section{Model odwróconego wahadła}
\begin{frame}[allowframebreaks]{Model dynamiki}
  \begin{align} \nonumber
    Q_w(q)\ddot{q}+
    \begin{bmatrix}
      Q_k && 0 \\
      0   && 0
    \end{bmatrix}
    \ddot{q}+
    Q_s(q)\ddot{q}
    +C_w(q, \dot{q})
    \dot{q}+C_k(q, \dot{q})\dot{q}\\ \nonumber 
    +C_s(q, \dot{q})\dot{q}
    +T(q)\dot{q}+D(q)=A^T(q)+Bu
  \end{align}, gdzie:
  \begin{description}
  	\item[$q=(x, y, \Theta, \phi_1, \phi_2, \alpha )^T$] --- wektor zmiennych stanu,
  	\item[$Q_w(q)$] --- macierz bezwładności wahadła,
  	\item[$Q_k$] --- macierz bezwładności kół,
  	\item[$Q_s$] --- macierz bezwładności silników,
  	\item[$C_w(q, \dot{q})$] --- macierz Coriolisa wahadła,
  	\item[$C_k(q, \dot{q})$] --- macierz Coriolisa kół,
  	\item[$C_s(q, \dot{q})$] --- macierz Coriolisa silników,
  	\item[$T$]               --- macierz tarcia pochodząca od silników,
  	\item[$D(q)$]            --- wektor grawitacji.
  \end{description}
  
\end{frame}

\begin{frame}{Model kinematyki}
	\begin{align}
	  \nonumber
		K_w = -\frac{1}{6}Ml^2(-3\dot{x}^2 +3\dot{\theta}\dot{x} l \sin{\theta}\sin{\alpha}
		-3\dot{\alpha}\dot{x}l \cos{\theta} \cos{\alpha} - \dot{\theta}^2l^2+\\
	  \nonumber
	    +\dot{\theta}^2 l^2 \cos{\alpha^2} - 3\dot{y}^2 -
	    3 \dot{\theta}\dot{y}l\cos{\theta}\sin{\alpha} - 3 \dot{\alpha}\dot{y}l
	    \sin{\theta}\cos{\alpha}-\dot{\alpha}^2
		)
	\end{align}
	\begin{figure}
		\includegraphics[width=0.4\textwidth]{img/wahadlo}
		\caption{Odwrócone wahadło na kołach --- robot balansujący}
	\end{figure}
\end{frame}

\section{Problem sterowania}
\begin{frame}{Przegląd możliwych rozwiązań (model zlinearyzowany w położeniu równowagi)}
Dla modelu zlinearyzowanego w położeniu równowagi można zastosować następujące regulatory.
	\begin{itemize}
		\item regulator typu PD, \pause
		\item regulator typu PID, \pause 
		\item regulator PID w kaskadzie z innym regulatorem, \pause
		\item inne rozwiązania.
	\end{itemize}
\end{frame}

\section{Budowa robota typu odwrócone wahadło}
	\begin{frame}{}
	W celu zbudowania robota typu odwrócone wahadło należy
	dobrać następujące elementy:
		\begin{itemize}
			\item czujniki,   \pause
			\item napędy,     \pause
			\item kontrolery napędów,  \pause
			\item pomiar prędkości obrotowej silników,    \pause
			\item jednostka centralna,   \pause
			\item moduł komunikacji.
		\end{itemize}
	\end{frame}


\section{}
\begin{frame}
  \begin{center}
    \huge
    Dziękujemy za uwagę!
  \end{center}
 

\end{frame}

\end{document}
